\documentclass[a4j]{jarticle}
\usepackage{graphicx}
\usepackage{ascmac}


\begin{document}

\title{計算機科学実験及演習2 \\ \bf ソフトウェア報告書1}
% ↓ここに自分の氏名を記入
\author{谷 勇輝 \\ \\入学年 平成27年 \\ 学籍番号 1029272870}
\西暦
\date{提出日: \today} % コンパイル時の日付が自動で挿入される
\maketitle

\section{課題1}
###
\subsection{実施内容}
\begin{description}
\item[(1)]~\\
ステージパラメータ(seed、難易度、敵の有無、その他地形等の変数)と既存エージェントを様々に変更し、動作を観察した。
\item[(2)]~\\
ch.idsia.agents.controllersパッケージ内に用意されたAgentプログラムのうち、
ForwardAgentについてソースコードと動作の比較からその内容を理解した。
\item[(3)]~\\
各Agentプログラムに頻繁に登場する isMarioAbleToJump と  isMarioOnGround の2つのパラメータについて理解を深めるため、
以下のソースコードをAgentプログラムの適切な箇所に挿入し、その内容について考察した。
\begin{screen}
System.out.println(isMarioAbleToJump + "/" + isMarioOnGround + " = " + action[Mario.KEY_JUMP]);
\end{screen}
\end{description}
###
\subsection{実行結果}
\begin{description}
\item[(1)]~\\
それぞれのパラメータについて、難易度、シードの値を様々に変更し観察し意味を理解した。
また、概ね自分の望むタイプのコースを用意できるようになった。
特記すべきパラメータについては以下に詳細を記述する。
	\begin{enumerate}
	\item Hill(丘)
	デフォルトで全ての難易度で出現する。\\
	下からのジャンプを透過し、床として使用できる。\\
	marioAIOptions.setHillStraightCountメソッドでfalseに設定することで出現しなくなる。
	\item Tubes(土管) ~\\
	デフォルトで全ての難易度で出現する。 パックンフラワーの出現率は難易度で異なるように思われる。\\
	marioAIOptions.setTubesCountメソッドでfalseに設定することで出現しなくなる。
	\item Gaps(落とし穴) ~\\
	難易度1以上で出現する。\\
	marioAIOptions.setGapsCountメソッドをfalseにすることで出現しなくなる。\\
	難易度0では値をtrueにしても出現しない。
	\item Cannons(砲台) ~\\
	難易度2以上で出現する。\\
	marioAIOptions.setCannonsCountメソッドをfalseにすることで出現しなくなる。\\
	難易度1以下では値をtrueにしても出現しない。
	\item DeadEnds(行き止まり) ~\\
	デフォルトでは出現しない。\\
	難易度に関わらず、marioAIOptions.setDeadEndsCountメソッドで有無を操作できる。\\
	以下のいずれか、もしくは複数の地形が発生する。\\
	\\
	 ・マリオがジャンプによって越えることのできない、地面からの壁\\
	 ・上方画面外から続く、空中をふさぐ壁。\\
	 ・上記の2つの地形に、その壁に密着するように延びる地面を追加した鍵型の地形。\\
	\\
	袋路を形成する可能性があり、後戻りが必要になりうる。
	\end{enumerate}
\item[(2)]~\\
マリオの動作を決定する配列について理解した。各エージェントプログラムの実装の趣旨と、その方法について理解を深めた。
\item[(3)]~\\
マリオの情報についての変数に対しての理解を深めた。
\end{description}
###
\subsection{結論と考察}
\begin{description}
\item[(1)]~\\
ステージパラメータについては、以下の三つのタイプのパラメータがあると分類することができる。
\begin{enumerate}
\item 難易度に関わらずデフォルトで出現するもの ~\\
コイン、ブロック、丘、土管
\item 特定難易度以上でしか出現しないもの
落とし穴、砲台
\item 値をtrueに設定した場合のみ出現するもの
行き止まり、Flat、隠しブロック (隠しブロックについては未検証)
\end{enumerate}

難易度とそれによるステージの変化については、続く課題を行う際に随時確認していきたい。
詳しく調査を行うことができたので、適宜適切なステージを構築し、人工知能プログラムの検証等に役立てたいと思う。

\item[(2)]~\\
マリオの動作を決定する配列について理解した。各エージェントプログラムの実装の趣旨と、その方法について理解を深めた。
\item[(3)]~\\
マリオの情報についての変数に対しての理解を深めた。
\end{description}

\newpage

\section{課題2}
\subsection{実施内容}
\subsection{実行結果}
\subsection{結論と考察}

%%%%%%%%%%%%%%%%%%%%%%%%%%%%%%%% 以下参考

\section{セクション}
内容

段落改行は一行あける
\subsection{サブセクション}
\subsection{サブセクション}

\subsection{箇条書き}
\begin{description}
\item [表題]~\\
内容はここに書く
\item[表題]\mbox{}\\
内容はここに書く
\item[表題に括弧を使う$<$括弧$>$]
\end{description}

\subsection{箇条書き(番号付き)}
\begin{enumerate}
\item 表題 ~\\
内容はここに書く 
\item 表題 \mbox{}\\
内容はここに書く 
\end{enumerate}

\subsection{図の挿入}
% 図のファイル名,拡大縮小率を調整する.
%\begin{center}
%  \includegraphics[scale=0.65]{figure01.eps}
%\end{center}

\subsection{スクリーン}
\begin{screen}
\\
単純改行\\
\\
$【括弧】$\\
$<括弧>$ \\
\end{screen}

\end{document}